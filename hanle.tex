\documentclass[12pt]{article}

\usepackage{fancyhdr}
\usepackage{geometry}
\usepackage{ucs}
\usepackage[utf8x]{inputenc}
\usepackage[T1]{fontenc}
\usepackage[ngerman]{babel}
\usepackage{amsmath,amssymb,amstext}
\usepackage{hyperref}
\usepackage{cancel}
\usepackage{dsfont}
\usepackage{physics}
\usepackage{lmodern}
\usepackage{enumerate}
\usepackage{enumitem}
\usepackage{graphicx}
\usepackage{listings, color}
\usepackage[labelfont=bf]{caption}
\usepackage{titling}

\lstset{basicstyle=\scriptsize} %Quellcode mit Umlauten und ganz klein
\lstset{literate=
  {Ö}{{\"O}}1
  {Ä}{{\"A}}1
  {Ü}{{\"U}}1
  {ß}{{\ss}}2
  {ü}{{\"u}}1
  {ä}{{\"a}}1
  {ö}{{\"o}}1
}


%Geometrie----------------------------------------------------------------------------------------------------------

\geometry{a4paper, top=25mm, left=15mm, right=15mm, bottom=25mm,headsep=10mm, footskip=10mm}
\pagestyle{fancy}
\setlength{\parindent}{0pt} %Zeileneinrückung

\fancyhf{} %Setzt voreingestellte Kopf-und Fußzeilen-Eigenschaften zurück

\lhead{\nouppercase{\leftmark}}
\chead{}
\rhead{\thepage}

\lfoot{}
\cfoot{}
\rfoot{}

\title{\vspace{0cm}{\Huge Fortgeschrittenen-Praktikum I:\\ \vspace{1cm} Der Hanle-Effekt}}
\author{Saskia Bondza\\Simon Stephan}
\date{durchgeführt am 28. und 29.09.2016}

\pretitle{%
  \begin{center}
  \LARGE
  \includegraphics[width=6cm,]{figures/siegel}\\[\bigskipamount]
}
\posttitle{\end{center}}

%neue Commands----------------------------------------------------------------------------------------------------------
\newcommand{\nab}{\vec{\nabla}} %direkter Befehl mit Vektorpfeil
\newcommand{\gra}[3][0.7]{
	\begin{minipage}[h!]{\textwidth}
		\centering
		\includegraphics[width=#1\textwidth]{figures/#2.png}
		\captionof{figure}{#3}
	\end{minipage}
	\vskip 30 pt
}
\newcommand{\graTwo}[4][0.5]{
	\begin{minipage}[h!]{\textwidth}
		\centering
		\includegraphics[width=#1\textwidth]{figures/#2.png}
		\includegraphics[width=#1\textwidth]{figures/#3.png}
		\captionof{figure}{#4}
	\end{minipage}
	\vskip 30 pt
}
\newcommand{\del}[2][]{\frac{\partial #1}{\partial #2}}
\newcommand{\code}[1]{\texttt{#1}}


%Titel,Inhalt----------------------------------------------------------------------------------------------------------

\begin{document}
\pagenumbering{gobble} %verstecke Seitenzahl
\maketitle
\newpage

\section*{Abstract}



\newpage

\thispagestyle{empty}
\tableofcontents
\newpage

%Schreiben----------------------------------------------------------------------------------------------------------
\pagenumbering{arabic} %verstecke Seitenzahl
\section{Einleitung}


\newpage
\section{Theoretische Grundlagen}


\newpage
\section{Versuchsaufbau und -Durchführung}


\newpage
\section{Auswertung}
\subsection{Zeit-Magnetfeld-Kalibration}
Wir haben die Intensitätsverläufe in Abhängigkeit der Zeit aufgenommen. Um diese in die magnetische Flussdichte $\vec B$ umrechnen zu können, benutzen wir die ebenfalls aufgenommenen Werte der Spannung an der Helmholtzspule, welche das Magnetfeld in z-Richtung $B_z$ erzeugt. Um daraus das Magnetfeld zu berechnen nutzen wir folgende Umrechnungsfaktoren\textsuperscript{\cite{anleitung}}:
\begin{align*}
	I&=U\\
	B_z&=3.363\cdot10^{-4}\cdot I
\end{align*}
Aus einer linearen Regression (siehe Abbildung \ref{zeitfit}) bestimmen wir die Umrechnung: $$U=a\cdot t+b$$ Also erhalten wir das Magnetfeld über: $$B_z=3.363\cdot10^{-4}\cdot(a\cdot t+b)$$

\gra{magnetfeld-zeit-kalibration}{Beispielhafter Datensatz zur linearen Regression. Es wurde zur Regression nur der linear ansteigende Bereich verwendet.\label{zeitfit}}
\subsection{Lorentz-Fits}
Da die 
\gra[1]{warm0-5}{Beispielhafter Lorentz-Fit}


\newpage
\section{Zusammenfassung und Diskussion}


\newpage
\section{Anhang}

%\subsection{Grafiken}


\subsection{Tabellen}
In den folgenden Tabellen \ref{cold0} - \ref{warm90} sind die Daten der Lorentz-Fits zu finden:
\begin{table}[h!]
\centering
\begin{tabular}{|c|c|c|c|c|}
\hline
$T/K$&$D/V$&$C/V$&$\omega/Hz$&$\tau/s$\\\hline\hline
$291.15$&$(-22.3\pm1.1)$&$(130\pm2)$&$(4320000\pm70000)$&$(1.14e-07\pm3e-09)$\\\hline
$290.15$&$(-22.5\pm1)$&$(135\pm2)$&$(4250000\pm60000)$&$(1.21e-07\pm3e-09)$\\\hline
$289.15$&$(-43.4\pm1.1)$&$(153\pm2)$&$(4300000\pm60000)$&$(1.14e-07\pm3e-09)$\\\hline
$288.15$&$(-42.1\pm1.1)$&$(166\pm2)$&$(4310000\pm50000)$&$(1.22e-07\pm2e-09)$\\\hline
$287.15$&$(-49.7\pm1.1)$&$(199\pm2)$&$(4220000\pm40000)$&$(1.25e-07\pm2e-09)$\\\hline
$286.15$&$(-73.7\pm1)$&$(216\pm2)$&$(4470000\pm40000)$&$(1.181e-07\pm1.7e-09)$\\\hline
$285.15$&$(-36.6\pm1)$&$(226\pm2)$&$(4410000\pm40000)$&$(1.111e-07\pm1.6e-09)$\\\hline
$284.15$&$(-70\pm1)$&$(236\pm2)$&$(4420000\pm40000)$&$(1.159e-07\pm1.6e-09)$\\\hline
$283.15$&$(-98.1\pm1)$&$(244\pm1.9)$&$(4650000\pm30000)$&$(1.18e-07\pm1.5e-09)$\\\hline
$282.15$&$(-91.3\pm1)$&$(258.6\pm1.9)$&$(4450000\pm30000)$&$(1.2e-07\pm1.4e-09)$\\\hline
$281.15$&$(-105.3\pm1.1)$&$(264\pm2)$&$(4510000\pm30000)$&$(1.298e-07\pm1.7e-09)$\\\hline
$280.15$&$(-95.6\pm1.1)$&$(279\pm2)$&$(4470000\pm30000)$&$(1.206e-07\pm1.4e-09)$\\\hline
$279.15$&$(-126.8\pm1)$&$(291\pm2)$&$(4490000\pm30000)$&$(1.195e-07\pm1.3e-09)$\\\hline
$277.15$&$(-115.3\pm1.1)$&$(321\pm2)$&$(4540000\pm30000)$&$(1.153e-07\pm1.2e-09)$\\\hline
$276.15$&$(-115.3\pm1)$&$(327\pm2)$&$(4510000\pm30000)$&$(1.16e-07\pm1.1e-09)$\\\hline
$275.15$&$(-129\pm1.1)$&$(343\pm2)$&$(4460000\pm30000)$&$(1.159e-07\pm1.2e-09)$\\\hline
$274.15$&$(-144.6\pm1.1)$&$(347\pm2)$&$(4420000\pm30000)$&$(1.189e-07\pm1.2e-09)$\\\hline
$272.15$&$(-116.1\pm1.1)$&$(361\pm2)$&$(4540000\pm30000)$&$(1.12e-07\pm1e-09)$\\\hline
\end{tabular}
\caption{Abkühlvorgang, Polarisation: 0°}
\end{table}

\begin{table}[h!]
\centering
\begin{tabular}{|c|c|c|c|c|}
\hline
$T/K$&$D/V$&$C/V$&$\omega/Hz$&$\tau/s$\\\hline\hline
$291.15$&$(6.36\pm0.18)$&$(-68\pm1.8)$&$(4390000\pm130000)$&$(5.04e-08\pm1.3e-09)$\\\hline
$290.15$&$(16.74\pm0.16)$&$(-80\pm1.5)$&$(5060000\pm1e+05)$&$(4.6e-08\pm9e-10)$\\\hline
$289.15$&$(33.92\pm0.18)$&$(-93.3\pm1.7)$&$(4750000\pm90000)$&$(4.84e-08\pm9e-10)$\\\hline
$288.15$&$(13\pm0.2)$&$(-105.1\pm1.8)$&$(4880000\pm110000)$&$(3.76e-08\pm6e-10)$\\\hline
$287.15$&$(25.9\pm0.2)$&$(-119.3\pm1.8)$&$(4720000\pm90000)$&$(4.26e-08\pm7e-10)$\\\hline
$286.15$&$(8.6\pm0.2)$&$(-127\pm1.8)$&$(4730000\pm90000)$&$(3.71e-08\pm5e-10)$\\\hline
$285.15$&$(43\pm0.2)$&$(-132.3\pm1.8)$&$(4660000\pm80000)$&$(4.09e-08\pm6e-10)$\\\hline
$284.15$&$(50.5\pm0.2)$&$(-146.6\pm1.9)$&$(4840000\pm80000)$&$(4.17e-08\pm5e-10)$\\\hline
$283.15$&$(10.4\pm0.2)$&$(-153\pm2)$&$(4860000\pm70000)$&$(4.57e-08\pm6e-10)$\\\hline
$282.15$&$(49.3\pm0.3)$&$(-162\pm2)$&$(4990000\pm90000)$&$(3.84e-08\pm5e-10)$\\\hline
$281.15$&$(15.4\pm0.2)$&$(-172\pm2)$&$(4770000\pm70000)$&$(4.32e-08\pm6e-10)$\\\hline
$280.15$&$(20\pm0.3)$&$(-175\pm2)$&$(4870000\pm80000)$&$(4.18e-08\pm6e-10)$\\\hline
$279.15$&$(64.5\pm0.3)$&$(-172\pm2)$&$(3910000\pm90000)$&$(3.76e-08\pm5e-10)$\\\hline
$277.15$&$(56.6\pm0.3)$&$(-197\pm2)$&$(5060000\pm80000)$&$(3.84e-08\pm5e-10)$\\\hline
$276.15$&$(20.7\pm0.3)$&$(-212\pm2)$&$(4870000\pm80000)$&$(3.8e-08\pm4e-10)$\\\hline
$275.15$&$(16.5\pm0.3)$&$(-217\pm3)$&$(4890000\pm80000)$&$(3.71e-08\pm4e-10)$\\\hline
$274.15$&$(44.5\pm0.3)$&$(-225\pm3)$&$(4990000\pm80000)$&$(3.79e-08\pm5e-10)$\\\hline
$272.15$&$(28\pm0.4)$&$(-225\pm3)$&$(5050000\pm1e+05)$&$(3.3e-08\pm4e-10)$\\\hline
\end{tabular}
\caption{Abkühlvorgang, Polarisation: 45°}
\end{table}

\begin{table}[h!]
\scriptsize\centering
\begin{tabular}{|c|l|l|l|l|}
\hline
$T/K$&$D/V$&$C/V$&$\omega/MHz$&$\tau/ns$\\\hline\hline
$291.15$&$76.2\pm1$&$-62.1\pm1.9$&$4.08\pm0.13$&$120\pm6$\\\hline
$290.15$&$95.8\pm1.2$&$-64\pm2$&$4.27\pm0.09$&$200\pm11$\\\hline
$289.15$&$79.5\pm1$&$-80.3\pm1.9$&$4.21\pm0.08$&$143\pm5$\\\hline
$288.15$&$113.7\pm1.1$&$-106\pm2$&$4.36\pm0.08$&$136\pm4$\\\hline
$287.15$&$99.1\pm0.9$&$-116.1\pm1.8$&$4.77\pm0.06$&$125\pm3$\\\hline
$286.15$&$72.4\pm1$&$-121.5\pm1.9$&$4.47\pm0.06$&$127\pm3$\\\hline
$285.15$&$103.8\pm1$&$-129.3\pm1.9$&$4.45\pm0.05$&$139\pm3$\\\hline
$284.15$&$85.9\pm1$&$-133.1\pm1.9$&$4.48\pm0.05$&$133\pm3$\\\hline
$283.15$&$106.8\pm1$&$-144.4\pm1.9$&$4.47\pm0.05$&$134\pm3$\\\hline
$282.15$&$156.6\pm0.9$&$-153.3\pm1.8$&$4.54\pm0.04$&$140\pm3$\\\hline
$281.15$&$129.4\pm1$&$-162.6\pm1.9$&$4.47\pm0.04$&$134\pm2$\\\hline
$280.15$&$95.8\pm1$&$-172.7\pm1.9$&$4.65\pm0.04$&$131\pm2$\\\hline
$279.15$&$152.7\pm1$&$-191.1\pm1.9$&$4.58\pm0.04$&$125\pm2$\\\hline
$277.15$&$127.9\pm1$&$-197\pm2$&$4.59\pm0.04$&$137\pm2$\\\hline
$276.15$&$137.4\pm1$&$-215.2\pm1.9$&$4.63\pm0.03$&$127.5\pm1.7$\\\hline
$275.15$&$146.1\pm0.9$&$-219.2\pm1.8$&$4.57\pm0.03$&$130\pm1.7$\\\hline
$274.15$&$135.6\pm1.7$&$-229\pm3$&$4.41\pm0.06$&$128\pm3$\\\hline
$272.15$&$44.7\pm1$&$-241\pm2$&$4.56\pm0.03$&$121.9\pm1.6$\\\hline
\end{tabular}
\caption{Abkühlvorgang, Polarisation: 90°\label{cold90}}
\end{table}

\begin{table}[h!]
\footnotesize\centering
\begin{tabular}{|c||l|l|l|l||c|}
\hline
$T/K$&$D/V$&$C/V$&$\omega/MHz$&$\tau/ns$&$\chi^2$ / ndf\\\hline\hline
$276.15$&$-675\pm2$&$1515\pm5$&$4.97\pm0.012$&$122\pm0.6$&$2.12221610270693$\\\hline
$277.15$&$-719\pm2$&$1486\pm5$&$5.026\pm0.013$&$122.7\pm0.6$&$6.37334384048335$\\\hline
$278.15$&$-697\pm4$&$1478\pm7$&$4.99\pm0.02$&$118.6\pm0.9$&$10.3285196763485$\\\hline
$279.15$&$-679\pm2$&$1441\pm5$&$5.02\pm0.013$&$121.1\pm0.6$&$3.76278489843081$\\\hline
$280.15$&$-664\pm2$&$1396\pm5$&$4.975\pm0.013$&$125.5\pm0.7$&$4.84694898649445$\\\hline
$281.15$&$-642\pm2$&$1378\pm5$&$4.919\pm0.013$&$124.5\pm0.6$&$3.44562802864485$\\\hline
$282.15$&$-611\pm4$&$1333\pm8$&$5.06\pm0.03$&$116.7\pm1.1$&$11.1317084108329$\\\hline
$283.15$&$-641\pm2$&$1308\pm5$&$4.923\pm0.015$&$121.1\pm0.7$&$9.78056166232292$\\\hline
$284.15$&$-574\pm2$&$1264\pm5$&$4.853\pm0.015$&$124.5\pm0.7$&$3.01675401610343$\\\hline
$285.15$&$-528\pm2$&$1224\pm5$&$5.021\pm0.015$&$122.7\pm0.7$&$2.1356340761012$\\\hline
$286.15$&$-481\pm2$&$1208\pm4$&$4.922\pm0.015$&$124.7\pm0.7$&$1.50552321386616$\\\hline
$287.15$&$-550\pm2$&$1175\pm5$&$4.982\pm0.016$&$121.5\pm0.7$&$4.34476357272359$\\\hline
$288.15$&$-538\pm2$&$1105\pm5$&$4.869\pm0.017$&$124.6\pm0.8$&$9.77806773859677$\\\hline
$289.15$&$-488\pm2$&$1070\pm4$&$5.034\pm0.017$&$121.8\pm0.8$&$3.65073548824493$\\\hline
$290.15$&$-442\pm2$&$1032\pm5$&$5\pm0.019$&$116.8\pm0.8$&$2.76656519607781$\\\hline
\end{tabular}
\caption{Aufwärmvorgang, Polarisation: 0°\label{warm0}}
\end{table}

\begin{table}[h!]
\centering
\begin{tabular}{|c|c|c|c|c|}
\hline
$T/K$&$D/V$&$C/V$&$\omega/Hz$&$\tau/s$\\\hline\hline
$276.15$&$(15.7\pm1.2)$&$(-1018\pm11)$&$(5160000\pm60000)$&$(4.06e-08\pm4e-10)$\\\hline
$277.15$&$(-2.5\pm1.2)$&$(-999\pm10)$&$(5330000\pm70000)$&$(3.94e-08\pm4e-10)$\\\hline
$278.15$&$(-21.2\pm1.2)$&$(-981\pm10)$&$(5260000\pm60000)$&$(3.99e-08\pm4e-10)$\\\hline
$279.15$&$(4.4\pm1.1)$&$(-956\pm10)$&$(5240000\pm60000)$&$(4.32e-08\pm4e-10)$\\\hline
$280.15$&$(-20\pm1.2)$&$(-923\pm10)$&$(5280000\pm70000)$&$(3.91e-08\pm4e-10)$\\\hline
$281.15$&$(-10.6\pm1.1)$&$(-910\pm10)$&$(5330000\pm70000)$&$(4.06e-08\pm4e-10)$\\\hline
$282.15$&$(8.9\pm1)$&$(-888\pm9)$&$(5290000\pm60000)$&$(4.04e-08\pm4e-10)$\\\hline
$283.15$&$(-23.9\pm1)$&$(-852\pm9)$&$(5300000\pm60000)$&$(4.03e-08\pm4e-10)$\\\hline
$284.15$&$(-3.9\pm0.9)$&$(-823\pm8)$&$(5480000\pm60000)$&$(4.12e-08\pm4e-10)$\\\hline
$285.15$&$(19.5\pm1)$&$(-775\pm9)$&$(5290000\pm70000)$&$(3.84e-08\pm4e-10)$\\\hline
$286.15$&$(9.7\pm1)$&$(-774\pm9)$&$(5250000\pm70000)$&$(3.98e-08\pm5e-10)$\\\hline
$287.15$&$(-15.1\pm0.9)$&$(-751\pm8)$&$(5250000\pm60000)$&$(4.02e-08\pm4e-10)$\\\hline
$288.15$&$(-19.4\pm0.9)$&$(-695\pm8)$&$(5380000\pm70000)$&$(3.87e-08\pm4e-10)$\\\hline
$289.15$&$(-39.5\pm0.9)$&$(-673\pm7)$&$(5340000\pm70000)$&$(3.93e-08\pm4e-10)$\\\hline
$290.15$&$(-26.5\pm0.9)$&$(-619\pm8)$&$(5410000\pm80000)$&$(3.62e-08\pm4e-10)$\\\hline
\end{tabular}
\caption{Aufwärmvorgang, Polarisation: 45°}
\end{table}

\begin{table}[h!]
\centering
\begin{tabular}{|c|c|c|c|c|}
\hline
$T/K$&$D/V$&$C/V$&$\omega/Hz$&$\tau/s$\\\hline\hline
$276.15$&$(586\pm2)$&$(-1085\pm4)$&$(5117000\pm16000)$&$(1.3e-07\pm8e-10)$\\\hline
$277.15$&$(541\pm2)$&$(-1054\pm5)$&$(4989000\pm17000)$&$(1.327e-07\pm9e-10)$\\\hline
$278.15$&$(581\pm2)$&$(-1036\pm4)$&$(5024000\pm17000)$&$(1.284e-07\pm8e-10)$\\\hline
$279.15$&$(535\pm2)$&$(-1002\pm5)$&$(5046000\pm18000)$&$(1.265e-07\pm9e-10)$\\\hline
$280.15$&$(517\pm2)$&$(-978\pm5)$&$(5033000\pm18000)$&$(1.281e-07\pm9e-10)$\\\hline
$281.15$&$(521\pm2)$&$(-946\pm4)$&$(5186000\pm18000)$&$(1.32e-07\pm1e-09)$\\\hline
$282.15$&$(508\pm2)$&$(-910\pm4)$&$(5059000\pm19000)$&$(1.306e-07\pm1e-09)$\\\hline
$283.15$&$(474\pm2)$&$(-865\pm4)$&$(4960000\pm20000)$&$(1.266e-07\pm1e-09)$\\\hline
$284.15$&$(453\pm2)$&$(-832\pm5)$&$(5e+06\pm20000)$&$(1.351e-07\pm1.1e-09)$\\\hline
$285.15$&$(475\pm2)$&$(-805\pm5)$&$(4950000\pm20000)$&$(1.302e-07\pm1.1e-09)$\\\hline
$286.15$&$(421\pm2)$&$(-771\pm4)$&$(5030000\pm20000)$&$(1.342e-07\pm1.2e-09)$\\\hline
$287.15$&$(391\pm2)$&$(-728\pm5)$&$(4900000\pm20000)$&$(1.33e-07\pm1.3e-09)$\\\hline
$288.15$&$(395\pm2)$&$(-693\pm4)$&$(5180000\pm30000)$&$(1.259e-07\pm1.3e-09)$\\\hline
$289.15$&$(366\pm2)$&$(-662\pm4)$&$(4830000\pm30000)$&$(1.293e-07\pm1.4e-09)$\\\hline
\end{tabular}
\caption{Aufwärmvorgang, Polarisation: 90°}
\end{table}


%\subsubsection{$\alpha$-Plateau Samarium}
%\lstinputlisting[language=MATLAB]{Rohdaten/alphaPlateau_Sm.txt}


%\newpage
%\subsection{Quellcode (MATLAB)}
%\lstinputlisting[language=MATLAB]{Rohdaten/alpha.m}
\clearpage

\subsection{Laborheft}
%\begin{minipage}{\textwidth}
%\centering
%\includegraphics[width=0.9\textwidth]{figures/IMG_20151002_141014.jpg}
%\end{minipage}

\newpage
\listoffigures

%Literatur----------------------------------------------------------------------------------------------------------

%\cite{les}
\newpage
\thispagestyle{empty}
\begin{thebibliography}{9}

%\bibitem{staat}
%  Tobijas Kotyk,
%  \emph{Versuche zur Radioaktivität im Physikalischen Fortgeschrittenen Praktikum an der Albert-Ludwigs-Universität Freiburg},
%  Albert-Ludwigs-Universität, Freiburg,
%  2005
  

  
%\bibitem{molmasse}
%  \emph{http://www.convertunits.com/molarmass/<ELEMENTNAME AUF ENGLISCH>}, Stand 28.09.2015
  

\bibitem{anleitung}
M. Köhli,
\emph{Versuchsanleitung Fortgeschrittenen Praktikum: Der Hanle-Effekt},
Albert-Ludwigs-Universität Freiburg,
2012

\bibitem{staat}
Wolf-Dieter Hasenclever,
\emph{Bau einer Apparatur zur Messung von Lebensdauern angeregter Atomzustände mit Hilfe des Hanle-Effekts},
Albert-Ludwigs-Universität Freiburg,
1970

\end{thebibliography}

\end{document}